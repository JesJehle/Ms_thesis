
With the decision of National Aeronautics and Space Administration (NASA), United States Geological Survey (USGS) and National Oceanic and Atmospheric Administration (NOAA), as well as European Space Agency (ESA) to provide open access to their satellite data, petabyte-scale archives of remote sensing data are now freely available.
The increasing availability, as well as the wide range of the spatial-, temporal-  and radiometric-resolution, have made remote sensing data, the best source of data for large-scale applications and studies in environmental system modelling.
Still, acquiring and preparing remote sensing data for large-scale applications is challenging and requires expertise in Geographic Information System (GIS) and access to high-performance computing resources.
Google Earth Engine is a cloud-based platform that strongly simplifies the access to high-performance computing resources while offering a multi-petabyte data catalogue of analysis-ready geospatial datasets.
By using the Earth Engine (EE) the acquisition and preprocessing of large-scale remote sensing data can strongly be simplified.

However, the EE API is accessed by a client library, whose application requires additional effort to learn. Furthermore, the client library is currently only available in JavaScript and Python. Therefore, scientists using the programming language R cannot apply the client library.

In the present work, a simplified access to capabilities of the EE for the R programming environment is presented.
The thesis aims to develop an R package - earthEngineGrabR - that simplifies the acquisition of remote sensing data by building an interface of R and EE. The interface enables to use EE as a backend-service to retrieve selected data products for a given region and time of interest in an analysis-ready format. Any acquiring and processing of the remote sensing data is entirely outsourced to EE and only the derived data products, are exported and imported into R. In a number of use cases, the earthEngineGrabR showed great potential to simplify and extend the users possibility to acquire remote sensing data for their research question. Compared to the standard approach of downloading and preprocessing the data on a local system, the earthEngineGrabR provided significant savings of time and resources while enabling the fast acquisition of processing intensive data products.


 
