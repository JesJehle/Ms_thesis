
\chapter{Conclusion}


In this master's thesis an R package that enables an interface between R and EE to retrieve selected data sets for a given region and time of interest, was successfully developed. The simplified excess to the capabilities of EE provided by the earthEngineGrabR packages showed great potential to strongly simplify the remote sensing data acquisition process. Compared to the standard approach to download and preprocess remote sensing data on a local system, the use of the earthEngineGrabR provided significant savings of time and processing resources in several use cases. 

However, the current version of the package yet presents a foundations. To be used in wide scientific context a further development and the implementation of additional features is necessary. 
The number of available data products and corresponding properties need to be extended.
Further, the implemented integration of R and Python is still error-prone and leads to operating system interoperability.  
Most of the limitations can be rectified with a further development of the package such as implementing additional features and improving the integration between R and Python.

Still, there are internal limitations of EE that restrict the possibility of the earthEngineGrabR. Altough EE can perform extensive computations, due to the export limit of EE to Google Drive, the download of large remote sensing datasets is efficient. This implies, instead of providing the acquisition of big geospatial data the earthEngineGrabR provides the processing of big geospatial data and the acquisition of the results of these computations.

To use the full potential of EE, the earthEngineGrabR has to provide comfortable and user-friendly access to the capabilities of EE to generate a variety of processing intensive data products and retrieve them in a strongly user defined format.



